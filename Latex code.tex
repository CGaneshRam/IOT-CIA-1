\documentclass{article}
\usepackage[a4paper, portrait, margin=1.1811in]{geometry}
\usepackage[english]{babel}
\usepackage[utf8]{inputenc}
\usepackage[T1]{fontenc}
\usepackage{helvet}
\usepackage{etoolbox}
\usepackage{graphicx}
\usepackage{titlesec}
\usepackage{caption}
\usepackage{booktabs}
\usepackage{xcolor} 
\usepackage[colorlinks, citecolor=cyan]{hyperref}
\usepackage{caption}
\captionsetup[figure]{name=Figure}
\graphicspath{ {./images/} }
\usepackage{scrextend}
\usepackage{fancyhdr}
\usepackage{graphicx}
\newcounter{lemma}
\newtheorem{lemma}{Lemma}
\newcounter{theorem}
\newtheorem{theorem}{Theorem}

\fancypagestyle{plain}{
	\fancyhf{}
	\renewcommand{\headrulewidth}{0pt}
	\renewcommand{\familydefault}{\sfdefault}
	
	\lhead{\color{cyan}\small \textbf{}\\ \color{black}
	\textit{}\\ }
	%\rhead{p-ISSN: 693-7554 \\ e-ISSN:2654-3990}
	%\rfoot{\thepage} --> Show the page number
	
}

%\pagestyle{plain}
\makeatletter
\patchcmd{\@maketitle}{\LARGE \@title}{\fontsize{16}{19.2}\selectfont\@title}{}{}
\makeatother

\usepackage{authblk}
\renewcommand\Authfont{\fontsize{10}{10.8}\selectfont}
\renewcommand\Affilfont{\fontsize{10}{10.8}\selectfont}
\renewcommand*{\Authsep}{, }
\renewcommand*{\Authand}{, }
\renewcommand*{\Authands}{, }
\setlength{\affilsep}{2em}  
\newsavebox\affbox

\titlespacing\section{0pt}{12pt plus 4pt minus 2pt}{0pt plus 2pt minus 2pt}
\titlespacing\subsection{12pt}{12pt plus 4pt minus 2pt}{0pt plus 2pt minus 2pt}
\titlespacing\subsubsection{12pt}{12pt plus 4pt minus 2pt}{0pt plus 2pt minus 2pt}


\titleformat{\section}{\normalfont\fontsize{10}{15}\bfseries}{\thesection.}{1em}{}
\titleformat{\subsection}{\normalfont\fontsize{10}{15}\bfseries}{\thesubsection.}{1em}{}
\titleformat{\subsubsection}{\normalfont\fontsize{10}{15}\bfseries}{\thesubsubsection.}{1em}{}

\titleformat{\author}{\normalfont\fontsize{10}{15}\bfseries}
{\thesection}{1em}{}
\title{\textbf{\huge Smart India- IOT in Agriculture}\\
	}
\date{}

\begin{document}

\pagestyle{headings}	
\newpage
\setcounter{page}{1}
\renewcommand{\thepage}{\arabic{page}}


\captionsetup[figure]{labelfont={bf},labelformat={default},labelsep=period,name={Figure }}	\captionsetup[table]{labelfont={bf},labelformat={default},labelsep=period,name={Table }}
\setlength{\parskip}{0.5em}

	
\maketitle

\section{Introduction}
Internet of Things (IoT) term represents a general concept 
for the ability of network devices to sense and collect data 
from around the world, and then share that data across the 
Internet where it can be processed and utilized for various 
interesting purposes. The IoT is comprised of smart 
machines interacting and communicating with other 
machines, objects, environments and infrastructures. Now a 
day‟s every persons are connected with each other using lots 
of communication way. Where most popular communication 
way is internet so in another word we can say internet which 
connect peoples. 

The essential idea of the Internet of Things (IoT) has been 
around for nearly two decades, and has attracted many 
researchers and industries because of its great estimated 
impact in improving our daily lives and society. When things 
like household appliances are connected to a network, they 
can work together in cooperation to provide the ideal service 
as a whole, not as a collection of independently working 
devices.This is useful for many of the real-world applications 
and services, and one would for example apply it to build a 
smart residence; windows can be closed automatically when 
the air conditioner is turned on, or can be opened for oxygen 
when the gas oven is turned on. The idea of IoT is especially 
valuable or persons with disabilities, as IoT technologies can 
support human activities at larger scale like building or 
society, as the devices can mutually cooperate to act as a 
total system.

The increasing global population demands improved production to provide food in all sectors, especially in agriculture. Smart agriculture is a better option for growing food production, resource management, and labour. This research provides an overview of predictive analysis, Internet of Things (IoT) devices with cloud management, security units for multi-culture.

The demand for food grain increases abruptly these years due to the fast-growing population. Food production should be improved for this reason in coming years globally. The IoT can also be applied to the agriculture sector with a wide range of sensors used for various smart agriculture targets. The conventional approach of agriculture is to enhance modernized cultivation with the exploration of the IoT region of interest in the agricultural field. 

Connecting multiple interconnected devices, such as several sensors, drivers and smart objects, to mobile devices through the Internet is a key element for the integration of scalable software, hardware and services for smart farming. With multiple sensors and described in green nature, the IoT can smartly build agriculture. The cost is very reasonable for all farming solutions with IoT-based smart agriculture.

\section{Ideas from the Author}
The key contributions from the author in this paper are:

1. More focused state-of-the-art research work has been identified in the field of IoT agriculture.

2. Characterize the existing IoT agriculture applications, sensors/devices, and communication protocols.

3. Proposed a taxonomy that further highlights the adopted IoT agriculture methods and approaches.

4. An IoT-based smart farming framework has been proposed that consists of basic IoT agriculture terms to identify the existing IoT solutions for the purpose of smart farming.

5. Identify the research gaps in terms of challenges and open issues.

\section{My Views}
Today’s agriculture is in a race. Farmers have to grow more products in deteriorating soil, declining land availability and increasing weather fluctuation. IoT-enabled agriculture allows farmers to monitor their product and conditions in real-time. They get insights fast, can predict issues before they happen and make informed decisions on how to avoid them. Additionally, IoT solutions in agriculture introduce automation, for example, demand-based irrigation, fertilizing and robot harvesting.

By the time we have 9 billion people on the planet, 70 percent of them will live in cities and towns. IoT-based greenhouses and hydroponic systems enable short food supply chains and should be able to feed the people. Smart closed-cycle agricultural systems allow growing food basically everywhere—in supermarkets, on skyscrapers’ walls and rooftops, in shipping containers and, of course, in the comfort of everyone’s home.


Plenty of ag IoT solutions are focused on optimizing the use of resources—water, energy, land. Precision farming using IoT relies on the data collected from diverse sensors in the field which helps farmers accurately allocate just enough resources to within one plant.

Not only do IoT-based systems for precision farming help producers save water and energy and, thus, make farming greener, but also significantly scale down on the use of pesticides and fertilizer. This approach allows getting a cleaner and more organic final product compared to traditional agricultural methods.

One of the benefits of using IoT in agriculture is the increased agility of the processes. Thanks to real-time monitoring and prediction systems, farmers can quickly respond to any significant change in weather, humidity, air quality as well as the health of each crop or soil in the field. In the conditions of extreme weather changes, new capabilities help agriculture professionals save the crops.

Data-driven agriculture helps both grow more and better products. Using soil and crop sensors, aerial drone monitoring and farm mapping, farmers better understand detailed dependencies between the conditions and the quality of the crops. Using connected systems, they can recreate the best conditions and increase the nutritional value of the products.


\section{Agreements}
1. "The IoT is a key element for the integration of scalable software, hardware, cost-effective process, self-sustainable, and smart decision for smart farming." Yes it is true, as The smart farming solution operates through a data-driven approach via sensors, where the data is turned into user-friendly information, enabling the managers to make quicker and wiser decisions related to their farm business.

2. "Cloud computing enhances forest cultivation by environmental analysis." This statement is true as cloud computing is becoming a powerful architecture to 
perform large-scale and complex computing, and has revolutionized the way that computing 
infrastructure is abstracted and used. In addition, an important goal of these technologies is to 
deliver computing as a solution for tackling big data, such as large-scale, multi-media and high 
dimensional data sets

3. "Laxmi C. Gavade et.al suggested a model to 
detect various features for greenfield such as soil, temperature, and the direction of sunlight with 
the assistance of sensors. Therefore, the productivity from the greenfield will be increased." This is a good idea as the conditions can be optimized and the plants can be grown in the fastest, healthiest and in the most optimum way.

\section{Disagreements}
1. "In the overall situation, the cost is very reasonable for all farming solutions with IoT based smart agriculture." I disagree with this statement as even though cost is reasonable, it is a question whether the farmers can afford it as they live a simple lifestyle.

2. "Since wireless communication solves many problems of a wired communication system. It also focuses on the former effort and money-saving that yields by optimizing water to be used in the green field." This statement is a contradiction to itself, though money is saved by optimizing water to be used in green field, wireless communication setup is more expensive than the wired one.




\begin{verbatim}


SUBMITTED BY:   C Ganesh Ram
                21011101034
                B Tech AI&DS-'A'(2nd year)
                SNU Chennai

\end{verbatim}

\end{document}